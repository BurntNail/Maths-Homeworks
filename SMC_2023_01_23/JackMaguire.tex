\documentclass{article}
\usepackage[utf8]{inputenc} % Required for inputting international characters
\usepackage[T1]{fontenc} % Output font encoding for international characters

\input{preamble}
\input{macros}

\graphicspath{{./images/}}

\title{\huge{9I}}
\author{\huge{Jack Maguire}}
\date{}

% This really annoys me:
% This is what all of the qs ask me to do

% 1 - find 2d and check for ranges x6
% 2 - find 2d and check for ranges
% 3 - find 2d and solve
% 4 - find 2d and solve
% 5 - find d and solve; find 2d and solve
% 6 - find d and solve; find 2d and solve
% 7 - read a graph
% 8 - find 2d and solve
% 9 - find 2d and solve

\begin{document}
\maketitle

\qs{	
	\begin{multicols}{2}
		\noindent
		\qsp{a}{
			\begin{align*}
				\differen &= 3x^2 - 6x + 1 \\
				\twodifferen &= 6x - 6 
			\end{align*}
			\begin{multicols}{2}
				\noindent
				\textbf{Concave}
				\begin{align*}
					6x - 6 &< 0 \\
					x &< 1
				\end{align*}	
				\columnbreak
				
				\textbf{Convex}
				\begin{align*}
					6x - 6 &> 0 \\
					x &> 1
				\end{align*}	
			\end{multicols}
		}
		\qsp{b}{
			\begin{align*}
				\differen &= 4x^3 - 9x^2 + 2 \\
				\twodifferen &= 12 x^2 - 18x
			\end{align*}
			\begin{multicols}{2}
				\noindent
				\textbf{Concave}
				\begin{align*}
					12x^2 - 18x &< 0 \\
					2x^2 - 3x &< 0 \\
					x(2x-3) &< 0 \\
					0 < x &< \frac{3}{2}
				\end{align*}	
				\columnbreak
				
				\textbf{Convex}
				\begin{align*}
					12x^2 - 18x &> 0 \\
					2x^2 - 3x &> 0 \\
					x(2x-3) &> 0 \\
					x < 0 \cup x &> \frac{3}{2}
				\end{align*}	
			\end{multicols}
		}
		\qsp{c}{
			\begin{align*}
				\differen &= \cos x \\
				\twodifferen &= - \sin x
			\end{align*}
			\begin{multicols}{2}
				\noindent
				\textbf{Concave}
				\begin{align*}
					- \sin x &< 0 \\
					0 < x &< \pi
				\end{align*}	
				\columnbreak
				
				\textbf{Convex}
				\begin{align*}
					- \sin x &> 0 \\
					\pi < x &> 2\pi
				\end{align*}	
			\end{multicols}
		}
		\columnbreak
		\qsp{d}{
			\begin{align*}
				\differen &= -2x + 3 \\
				\twodifferen &= -2
			\end{align*}
			\begin{multicols}{2}
				\noindent
				\textbf{Concave}
				\begin{align*}
					-2 &< 0 \\
					\text{Always Concave}
				\end{align*}	
				\columnbreak
				
				\textbf{Convex}
				\begin{align*}
					-2 &> 0 \\
					\text{Never Convex}
				\end{align*}	
			\end{multicols}
		}
		\qsp{e}{
			\begin{align*}
				\differen &= e^x - 2x \\
				\twodifferen &= e^x - 2
			\end{align*}
			\begin{multicols}{2}
				\noindent
				\textbf{Concave}
				\begin{align*}
					e^x - 2 &< 0 \\
					e^x &< 2 \\
					x &< \ln 2
				\end{align*}	
				\columnbreak
				
				\textbf{Convex}
				\begin{align*}
					e^x - 2 &> 0 \\
					e^x &> 2 \\
					x &> \ln 2
				\end{align*}	
			\end{multicols}
		}
		\qsp{f}{
			\begin{align*}
				\differen &= \frac{1}{x} \\
				\twodifferen &= \frac{-1}{x^2}
			\end{align*}
			\begin{multicols}{2}
				\noindent
				\textbf{Concave}
				\begin{align*}
					\frac{-1}{x^2} &< 0 \\
					-1 &< 0 \\
					\text{Always Concave}
				\end{align*}	
				\columnbreak
				
				\textbf{Convex}
				\begin{align*}
					\frac{-1}{x^2} &> 0 \\
					-1 &> 0 \\
					\text{Never Concave}
				\end{align*}	
			\end{multicols}
		}	
	\end{multicols}
}

\qs{
\qsp{a}{
\[ \sin^2 y + \cos^2 y = 1 \quad \therefore \quad \cos y = \sqrt{1 - \sin^2 y} \]
\[ y = \arcsin x \quad \therefore \quad x = \sin y \]
\begin{align*}
	1 &= \cos y \differen \\
	\differen &= \frac{1}{\cos y} \\ 
	&= \frac{1}{\sqrt{1 - \sin^2 y}} \\
	&= \frac{1}{\sqrt{1 - x^2}}
\end{align*}
}
\qsp{b}{
\begin{align*}
	\differen &= \left( 1 - x^2 \right)^{-\frac{1}{2}} \\
	\twodifferen &= - \frac{1}{2} \left( 1-x^2 \right)^{-\frac{3}{2}} (-2x) \\
	&= \frac{x}{2 \sqrt{1-x^2}^3}
\end{align*}

\lineyboi

\begin{align*}
	\frac{x}{2 \sqrt{1-x^2}^3} &< 0 \\
	x &< 0 \\
	-1 < x &< 0
\end{align*}
}
\qsp{c}{
\begin{align*}
	\frac{x}{2 \sqrt{1-x^2}^3} &> 0 \\
	x &> 0 \\
	0 < x &< 1
\end{align*}
}
\qsp{d}{
\begin{align*}
	x &= 0 \\
	y &= \arcsin x = \arcsin 0 \\
	&= 0
\end{align*}
\[ = (0, 0) \]
}
}


\qs{
\qsp{a}{
\begin{align*}
	\differen &= -2 \cos x \sin x - 2 \cos x \\
	\twodifferen &= -2 \left( - \sin^2 x + \cos^2 x \right) + 2 \sin x \\
	&= -2 \left( \cos^2 x - \sin^2 x \right) + 2 \sin x \\
\end{align*}
\lineyboi
\begin{align*}
	0 &= -2 \left( \cos^2 x - \sin^2 x \right) + 2 \sin x \\
	0 &= \cos^2 x - \sin^2x  - \sin x \\
	0 &= 1 - \sin^2x - \sin^2x - \sin x \\
	0 &= 2 \sin^2x + \sin x - 1 \\
	0 &= \left( \sin x + 1 \right) \left( 2 \sin x - 1 \right) \\
	\sin x &= -1, \frac{1}{2} \\
	x &= \frac{3}{2}\pi, \frac{1}{6} \pi, \frac{5}{6} \pi
\end{align*}
\lineyboi
\begin{itemize}
	\ii \( (\frac{3}{2}\pi, 2) \)
	\ii \( (\frac{1}{6}\pi, -\frac{1}{4}) \)
	\ii \( (\frac{5}{6}\pi, -\frac{1}{4}) \)
\end{itemize}
}
\qsp{b}{
\begin{align*}
	\differen &= -\frac{\left( 3x^2 - 4x + 1 \right) \left( x - 2 \right) - \left( x^3 - 2x^2 + x - 1 \right) }{\left( x - 2 \right)^2} \\
	&= -\frac{3x^3 -6x^2 -4x^2 +8x +x -2 -x^3 +2x^2 -x +1}{\left( x - 2 \right)^2} \\
	&= \frac{-2x^3 +8x^2 -8x +1}{x^2 - 4x + 4} \\
	&= \frac{-2x^3 +8x^2 -8x}{x^2 - 4x + 4} + \frac{1}{x^2 - 4x + 4} \\
	&= \frac{-2x(x^2 - 4x + 4)}{x^2 + 4x + 4} + \frac{1}{x^2 - 4x + 4} \\
	&= -2x + \frac{1}{x^2 - 4x + 4} \\
	\twodifferen &= -2 -\left(x^2 - 4x + 4\right)^{-2}(2x-4) \\
	&= \frac{4-2x}{(x-2)^4} - 2 \\
\end{align*}
\lineyboi
\begin{align*}
	\frac{4-2x}{(x-2)^4} - 2 &= 0 \\
	\frac{4-2x}{(x-2)^4} &= 2 \\
	2-x &= (x-2)^4 \\
	-(x-2) &= (x-2)^4 \\
	-1 &= (x-2)^3 \\
	x-2 &= -1 \\
	x &= 1 \\
\end{align*}
\lineyboi
\[ = (1, 1) \]
}
\qsp{c}{
\begin{align*}
	\differen &= \frac{2x^2\left( x^2 - 4 \right) - x^3 \left( 2x \right)}{\left(x^2 - 4\right)^2} \\
	&= \frac{3x^4 - 8x^2 - 2x^4}{\left(x^2 - 4\right)^2} \\
	&= \frac{x^4-12x^2}{\left(x^2 - 4\right)^2} \\
	\twodifferen &= \frac{\left( 4x^3 - 24x \right)\left( x^2-4 \right)^2 - 2\left( x^4 - 12x^2 \right)\left( x^2 - 4 \right) (2x)}{\left(x^2 - 4\right)^4} \\
	&= \frac{\left( 4x^3 - 24x \right)\left( x^2-4 \right) - 4x\left( x^4 - 12x^2 \right)}{\left(x^2 - 4\right)^3} \\
	&= \frac{4x^5 - 16x^3 - 24x^3 + 96x - 4x^5 + 48x^3}{\left(x^2 - 4\right)^3} \\
	&= \frac{8x^3 + 96x}{\left(x^2 - 4\right)^3} \\
	&= \frac{8x\left( x^2 + 12 \right)}{\left(x^2 - 4\right)^3} \\
\end{align*}
\lineyboi
\begin{align*}
	0 = \frac{8x\left( x^2 + 12 \right)}{\left(x^2 - 4\right)^3}
	0 &= 8x\left( x^2 + 12 \right) \\
	x &= 0 \\ \\
	0 &= x^2 + 12 \\
	& \text{Discard}
\end{align*}
}
\lineyboi
\[ = (0, 0) \]
}


\qs{
\begin{align*}
	\differen &= \frac{2x^2}{x} + 4x \ln x \\
	&= 2x + 4x \ln x \\
	&= 2x \left( 1 + 2\ln x \right) \\
	\twodifferen &= 2\left( 1 + 2\ln x \right) + 2x \frac{2}{x} \\
	&= 2 + 4 \ln x + 4 \\
	&= 6 + 4 \ln x \\
\end{align*}
\lineyboi
\begin{align*}
	0 &= 6 + 4 \ln x \\
	-\frac{3}{2} &= \ln x \\
	x &= e^{-\frac{3}{2}} \\
	x & 0.223
\end{align*}
\[ = (0.223, -0.149) \]
}

\qs{
\qsp{a}{
\begin{align*}
	\differen &= e^x \left( x^2 - 2x + 2 \right) + e^x \left( 2x - 2 \right)
	&= e^x x^2
\end{align*}
\lineyboi
\[ 	0 = e^x x^2 \]
\begin{multicols}{2}
	\noindent
	\begin{align*}
		0 &= e^x \\
		x &= \ln 0 \\
		& \text{Discard}
	\end{align*}
	\columnbreak
	\begin{align*}
		0 &= x^2 \\
		x &= 0
	\end{align*}
\end{multicols}
\lineyboi
\[ = (0, 2) \]
\begin{tabular}{r|l}
	\( f^\prime(-0.1) \) & \( 0.09 \) \\
	\( f^\prime(0.1) \) & \( 0.011 \) \\
\end{tabular}
Point of Inflection?
}

\qsp{b}{
\[ \twodifferen = e^x \left( x^2 + 2x \right) = e^xx(x+2) \]
\begin{multicols}{3}
	\noindent
	\begin{align*}
		0 &= e^x \\
		x &= \ln 0 \\
		& \text{Discard}
	\end{align*}
	\columnbreak
	\[ x = 0 \]
	\columnbreak
	\begin{align*}
		0 &= x + 2 \\
		x &= -2 \\
	\end{align*}
\end{multicols}
Since \( x = 0 \) is a stationary point, I only need \( x = 2 \).
\[ = (2, 2e^2)\]
}
}

\qs{
\begin{multicols}{2}
	\noindent
	\qsp{a}{
		\[ \differen = e^x(x+1) \]
		\lineyboi
		\[ 0 = e^x(x+1) \]
		\begin{multicols}{2}
			\noindent
			\begin{align*}
				0 &= e^x \\
				x &= \ln 0 \\
				& \text{Discard}
			\end{align*}
			\columnbreak
			\begin{align*}
				0 &= x + 1\\
				x &= -1 \\
			\end{align*}
		\end{multicols}
		\lineyboi
		\hfill
		\[ = (-1, \frac{-1}{e}) \]
		\begin{tabular}{r|l}
			\( f^\prime(-0.1) \) & \( -0.09 \) \\
			\( f^\prime(0.1) \) & \( 0.11 \) \\
		\end{tabular}
		Minimum Point
	}

	\columnbreak
	\qsp{b}{
		\[ \twodifferen = e^x \left( x+1 + 1 \right) =  e^x(x+2) \]
		\lineyboi
		\[ 0 = e^x (x+2) \]
		\begin{multicols}{2}
			\noindent
			\begin{align*}
				0 &= e^x \\
				x &= \ln 0 \\
				& \text{Discard}
			\end{align*}	
			\columnbreak
			\begin{align*}
				0 &= x + 2\\
				x &= -2 \\
			\end{align*}
		\end{multicols}
		\lineyboi
		\hfill
		\[ = (-2, \frac{-2}{e^2})) \]
	}
\end{multicols}
\qsp{c}{
\vspace{5cm}
}
}

\qs{
\begin{tabular}{r|cc}
	& \( f^\prime \) & \( f^{\prime\prime} \) \\
	\hline
	A & Negative & Positive \\
	B & Zero & Positive \\
	C & Positive & Negative \\
	D & Zero & Negative
\end{tabular}
}

\qs{
\begin{align*}
	\differen &= \sec^2 x \\
	\twodifferen &= 2 \tan x \sec^2 x
\end{align*}
\lineyboi
\begin{align*}
	0 &= 2 \tan x \sec^2 x \\
	0 &= \frac{2 \sin x}{\cos^3 x} \\
	0 &= 2 \sin x
	x &= \arcsin 0 \\
	x &= 0 
\end{align*}
\[ = (0, 0) \]
}

\qs{
\qsp{a}{
\begin{align*}
	\differen &= 15x(3x - 1)^4 + (3x-1)^5
	\twodifferen &= 180x^2 (3x-1)^3 + 15(3x-1)^4 + 15(3x-1)^4 \\
	&= (3x-1)^3 \left( 180x + 30(3x-1)  \right) \\
	&= (3x-1)^3(270x - 30)
\end{align*}
}
\qsp{b}{
\[ (3x-1)^3(270x - 30) \]
\begin{multicols}{2}
	\noindent
	\begin{align*}
		0 &= (3x-1)^3 \\
		0 &= 3x - 1 \\
		x &= \frac{1}{3}
	\end{align*}
	\columnbreak
	\begin{align*}
		0 &= 270x - 30 \\
		30 &= 270x \\
		x &= \frac{1}{9}
	\end{align*}
\end{multicols}
\[ = \left( \frac{1}{3}, 0 \right), \left( \frac{1}{9}, -\frac{32}{2187} \right) \]
}
}

\end{document}