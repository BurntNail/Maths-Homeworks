\documentclass{article}
\usepackage[utf8]{inputenc} % Required for inputting international characters
\usepackage[T1]{fontenc} % Output font encoding for international characters

%%%%%%%%%%%%%%%%%%%%%%%%%%%%%%%%%
% PACKAGE IMPORTS
%%%%%%%%%%%%%%%%%%%%%%%%%%%%%%%%%
\usepackage[tmargin=2cm,rmargin=1in,lmargin=1in,margin=0.85in,bmargin=2cm,footskip=.2in]{geometry}
\usepackage{amsmath,amsfonts,amsthm,amssymb,mathtools} %general maths stuff
\usepackage[varbb]{newpxmath} %maths tools
\usepackage{xfrac} %a/b fractions
\usepackage[makeroom]{cancel}
\usepackage{mathtools}
\usepackage{bookmark}
\usepackage{enumitem}
\usepackage{hyperref,theoremref}
\hypersetup{
	pdftitle={Maths Homework},
	colorlinks=true, linkcolor=black!90,
	bookmarksnumbered=true,
	bookmarksopen=true
}
\usepackage[most,many,breakable]{tcolorbox}
\usepackage{xcolor}
\usepackage{varwidth}
\usepackage{varwidth}
\usepackage{etoolbox}
\usepackage{nameref}
\usepackage{multicol,array}
\usepackage{tikz-cd}
\usepackage[ruled,vlined,linesnumbered]{algorithm2e}
\usepackage{comment} % enables the use of multi-line comments (\ifx \fi) 
\usepackage{import}
\usepackage{xifthen}
\usepackage{pdfpages}
\usepackage{transparent}
\usepackage{tikzsymbols}

\usepackage[default]{raleway}
\usepackage{sectsty}
\renewcommand*\familydefault{\sfdefault} % Force the sans-serif version of any font used
\allsectionsfont{\sffamily\mdseries\upshape} % (See the fntguide.pdf for font help)

\usepackage[nottoc,notlof,notlot]{tocbibind} % Put the bibliography in the ToC
\usepackage[titles,subfigure]{tocloft} % Alter the style of the Table of Contents
\renewcommand{\cftsecfont}{\rmfamily\mdseries\upshape}
\renewcommand{\cftsecpagefont}{\rmfamily\mdseries\upshape} % No bold!

\usepackage{calc}
\usepackage{subfig}
\usepackage{pgfplots}
\pgfplotsset{compat = newest}
\usepackage{siunitx}


%%%%%%%%%%%%%%%%%%%%%%%%%%%%%%J
% SELF MADE COLORS
%%%%%%%%%%%%%%%%%%%%%%%%%%%%%%

\definecolor{myg}{RGB}{56, 140, 70}
\definecolor{myb}{RGB}{45, 111, 177}
\definecolor{myr}{RGB}{199, 68, 64}
\definecolor{mytheorembg}{HTML}{F2F2F9}
\definecolor{mytheoremfr}{HTML}{00007B}
\definecolor{mylenmabg}{HTML}{FFFAF8}
\definecolor{mylenmafr}{HTML}{983b0f}
\definecolor{mypropbg}{HTML}{f2fbfc}
\definecolor{mypropfr}{HTML}{191971}
\definecolor{myexamplebg}{HTML}{F2FBF8}
\definecolor{myexamplefr}{HTML}{88D6D1}
\definecolor{myexampleti}{HTML}{2A7F7F}
\definecolor{mydefinitbg}{HTML}{E5E5FF}
\definecolor{mydefinitfr}{HTML}{3F3FA3}
\definecolor{notesgreen}{RGB}{0,162,0}
\definecolor{myp}{RGB}{197, 92, 212}
\definecolor{mygr}{HTML}{2C3338}
\definecolor{myred}{RGB}{127,0,0}
\definecolor{myyellow}{RGB}{169,121,69}
\definecolor{myexercisebg}{HTML}{F2FBF8}
\definecolor{myexercisefg}{HTML}{88D6D1}


%%%%%%%%%%%%%%%%%%%%%%%%%%%%
% TCOLORBOX SETUPS
%%%%%%%%%%%%%%%%%%%%%%%%%%%%
%================================
% EXAMPLE BOX
%================================

\newtcbtheorem[number within=section]{Example}{Example}
{%
	colback = myexamplebg
	,breakable
	,colframe = myexamplefr
	,coltitle = myexampleti
	,boxrule = 1pt
	,sharp corners
	,detach title
	,before upper=\tcbtitle\par\smallskip
	,fonttitle = \bfseries
	,description font = \mdseries
	,separator sign none
	,description delimiters parenthesis
}
{ex}

%================================
% Question BOX
%================================

\makeatletter
\newtcbtheorem{question}{Question}{enhanced,
	breakable,
	colback=gray!20!white,
	colframe=mygr,
	attach boxed title to top left={yshift*=-\tcboxedtitleheight},
	fonttitle=\bfseries,
	title={},
	boxed title size=title,
	boxed title style={%
		sharp corners,
		rounded corners=northwest,
		colback=tcbcolframe,
		boxrule=0pt,
	},
	underlay boxed title={%
		\path[fill=tcbcolframe] (title.south west)--(title.south east)
		to[out=0, in=180] ([xshift=5mm]title.east)--
		(title.center-|frame.east)
		[rounded corners=\kvtcb@arc] |-
		(frame.north) -| cycle;
	},
	#1
}{def}
\makeatother


%================================
% NOTE BOX
%================================


\usetikzlibrary{arrows,calc,shadows.blur}
\tcbuselibrary{skins}
\newtcolorbox{questionpart}[2][]{%
	enhanced jigsaw,
	colback=white,
	colframe=gray!80!black,
	size=small,
	boxrule=1pt,
	title=\textit{#2},
	halign title=flush center,
	coltitle=black,
	breakable,
	drop shadow=black!50!white,
	attach boxed title to top left={xshift=0.2cm,yshift=-\tcboxedtitleheight/2,yshifttext=-\tcboxedtitleheight/2},
	minipage boxed title=0.5cm,
	boxed title style={%
		colback=white,
		size=fbox,
		boxrule=1pt,
		boxsep=2pt,
		underlay={%
			\coordinate (dotA) at ($(interior.west) + (-0.5pt,0)$);
			\coordinate (dotB) at ($(interior.east) + (0.5pt,0)$);
			\begin{scope}
				\clip (interior.north west) rectangle ([xshift=3ex]interior.east);
				\filldraw [white, blur shadow={shadow opacity=60, shadow yshift=-.75ex}, rounded corners=2pt] (interior.north west) rectangle (interior.south east);
			\end{scope}
			\begin{scope}[gray!80!black]
				\fill (dotA) circle (2pt);
				\fill (dotB) circle (2pt);
			\end{scope}
		},
	},
	#1,
}

%%%%%%%%%%%%%%%%%%%%%%%%%%%%%%
% SELF MADE COMMANDS
%%%%%%%%%%%%%%%%%%%%%%%%%%%%%%

\newcommand{\qsn}[2]{\begin{question*}{#1}{}#2\end{question*}}
\newcommand{\qs}[1]{\begin{question}{}{}#1\end{question}}

\newcommand{\qsp}[2]{\begin{questionpart}{#1}{}#2\end{questionpart}}

\newcommand{\plotbasic}[3][black]{ % colour func formatted
	\begin{tikzpicture}
		\begin{axis}[
			axis lines = center,
			xlabel = \(x\),
			ylabel = \(f(x)\),
			]
			
			\addplot [
			domain=-5:5, 
			samples=100, 
			color={#1},
			]
			{#2};
			\addlegendentry{#3}
		\end{axis}
	\end{tikzpicture}
}
\newenvironment{plotter}{
	\begin{tikzpicture}
		\begin{axis}[
			axis lines = center,
			xlabel = \(x\),
			ylabel = \(f(x)\),
			]
		}
		{
		\end{axis}
	\end{tikzpicture}
}


\newenvironment{unitcircled}{
	\usetikzlibrary {intersections}
	\begin{tikzpicture}[scale=2]
		\draw[step=.5cm,gray,very thin] (-1.2,-1.2) grid (1.2,1.2);
		\draw[->] (-1.2,0) -- (1.2,0) coordinate (x axis);
		\draw[->] (0,-1.2) -- (0,1.2) coordinate (y axis);
		\draw (0,0) circle [radius=1cm];
		
		\foreach \x/\xtext in {-1, -0.5/-\frac{1}{2}, 0, 0.5/\frac{1}{2}, 1}
		\draw (\x cm,1pt) -- (\x cm,-1pt) node[anchor=north,fill=white] {$\xtext$};
		\foreach \y/\ytext in {-1, -0.5/-\frac{1}{2}, 0.5/\frac{1}{2}, 1}
		\draw (1pt,\y cm) -- (-1pt,\y cm) node[anchor=east,fill=white] {$\ytext$};
	} {
	\end{tikzpicture}
}

\newcommand*\circled[1]{\tikz[baseline=(char.base)]{
		\node[shape=circle,draw,inner sep=1pt] (char) {#1};}}
\newcommand\getcurrentref[1]{%
	\ifnumequal{\value{#1}}{0}
	{??}
	{\the\value{#1}}%
}
\newcommand{\getCurrentSectionNumber}{\getcurrentref{section}}


\newcounter{mylabelcounter}

\makeatletter
\newcommand{\setword}[2]{%
	\phantomsection
	#1\def\@currentlabel{\unexpanded{#1}}\label{#2}%
}
\makeatother

\tikzset{
	symbol/.style={
		draw=none,
		every to/.append style={
			edge node={node [sloped, allow upside down, auto=false]{$#1$}}}
	}
}


%%%%%%%%%%%%%%%%%%%%%%%%%%%%%%%%%%%%%%%%%%%
% TABLE OF CONTENTS
%%%%%%%%%%%%%%%%%%%%%%%%%%%%%%%%%%%%%%%%%%%

\usepackage{tikz}
\definecolor{doc}{RGB}{0,60,110}
\usepackage{titletoc}
\contentsmargin{0cm}
\titlecontents{chapter}[3.7pc]
{\addvspace{30pt}%
	\begin{tikzpicture}[remember picture, overlay]%
		\draw[fill=doc!60,draw=doc!60] (-7,-.1) rectangle (-0.9,.5);%
		\pgftext[left,x=-3.5cm,y=0.2cm]{\color{white}\Large\sc\bfseries Chapter\ \thecontentslabel};%
	\end{tikzpicture}\color{doc!60}\large\sc\bfseries}%
{}
{}
{\;\titlerule\;\large\sc\bfseries Page \thecontentspage
	\begin{tikzpicture}[remember picture, overlay]
		\draw[fill=doc!60,draw=doc!60] (2pt,0) rectangle (4,0.1pt);
\end{tikzpicture}}%
\titlecontents{section}[3.7pc]
{\addvspace{2pt}}
{\contentslabel[\thecontentslabel]{2pc}}
{}
{\hfill\small \thecontentspage}
[]
\titlecontents*{subsection}[3.7pc]
{\addvspace{-1pt}\small}
{}
{}
{\ --- \small\thecontentspage}
[ \textbullet\ ][]

\makeatletter
\renewcommand{\tableofcontents}{%
	\chapter*{%
		\vspace*{-20\p@}%
		\begin{tikzpicture}[remember picture, overlay]%
			\pgftext[right,x=15cm,y=0.2cm]{\color{doc!60}\Huge\sc\bfseries \contentsname};%
			\draw[fill=doc!60,draw=doc!60] (13,-.75) rectangle (20,1);%
			\clip (13,-.75) rectangle (20,1);
			\pgftext[right,x=15cm,y=0.2cm]{\color{white}\Huge\sc\bfseries \contentsname};%
	\end{tikzpicture}}%
	\@starttoc{toc}}
\makeatother
%Symbols
\newcommand{\dg}{^\circ}
\newcommand{\dang}{\measuredangle} %% Directed angle
\newcommand{\lm}{\lambda}
\newcommand{\uin}{\mathbin{\rotatebox[origin=c]{90}{$\in$}}}
\newcommand{\usubset}{\mathbin{\rotatebox[origin=c]{90}{$\subset$}}}

%Shortcuts
\newcommand{\ii}{\item}
\newcommand{\lthen}{\rightarrow}
\newcommand{\opname}{\operatorname}

%Diff/Int
\newcommand{\mandifferen}[2]{\frac{\Delta #1}{\Delta #2}}
\newcommand{\differen}[1][y]{\frac{\Delta #1}{\Delta x}}
\newcommand{\differend}[1][y]{\dfrac{\Delta #1}{\Delta x}}
\newcommand{\twodifferen}[1][y]{\frac{\Delta^2 #1}{\Delta x^2}}
\newcommand{\twodifferend}[1][y]{\dfrac{\Delta^2 #1}{\Delta x^2}}
\newcommand{\limit}[2]{\lim\limits_{#1 \to #2}}
\newcommand{\integratestart}[4][x]{\int_{#3}^{#4} \left( {#2} \right) d{#1}} %x/t/etc func lowerlim upperlim
\newcommand{\integrateinner}[3]{\left[ {#1} \right]_{#2}^{#3}} % func lowerlim upperlim

\DeclareMathOperator{\cosec}{cosec}
\DeclareMathOperator{\ud}{\textit{undefined}}
\DeclareMathOperator{\iqr}{IQR}

\newcommand{\func}[2][f]{#1 \left( #2 \right)}
\newcommand{\acb}[2]{{#1 \choose #2}}

\renewcommand\qedsymbol{QED}
\newcommand{\ans}{_{\slash \slash}}


%---------------------------------------
% BlackBoard Math Fonts :-
%---------------------------------------

%Captital Letters
\newcommand{\bbA}{\mathbb{A}}	\newcommand{\bbB}{\mathbb{B}}
\newcommand{\bbC}{\mathbb{C}}	\newcommand{\bbD}{\mathbb{D}}
\newcommand{\bbE}{\mathbb{E}}	\newcommand{\bbF}{\mathbb{F}}
\newcommand{\bbG}{\mathbb{G}}	\newcommand{\bbH}{\mathbb{H}}
\newcommand{\bbI}{\mathbb{I}}	\newcommand{\bbJ}{\mathbb{J}}
\newcommand{\bbK}{\mathbb{K}}	\newcommand{\bbL}{\mathbb{L}}
\newcommand{\bbM}{\mathbb{M}}	\newcommand{\bbN}{\mathbb{N}}
\newcommand{\bbO}{\mathbb{O}}	\newcommand{\bbP}{\mathbb{P}}
\newcommand{\bbQ}{\mathbb{Q}}	\newcommand{\bbR}{\mathbb{R}}
\newcommand{\bbS}{\mathbb{S}}	\newcommand{\bbT}{\mathbb{T}}
\newcommand{\bbU}{\mathbb{U}}	\newcommand{\bbV}{\mathbb{V}}
\newcommand{\bbW}{\mathbb{W}}	\newcommand{\bbX}{\mathbb{X}}
\newcommand{\bbY}{\mathbb{Y}}	\newcommand{\bbZ}{\mathbb{Z}}

%---------------------------------------
% MathCal Fonts :-
%---------------------------------------

%Captital Letters
\newcommand{\mcA}{\mathcal{A}}	\newcommand{\mcB}{\mathcal{B}}
\newcommand{\mcC}{\mathcal{C}}	\newcommand{\mcD}{\mathcal{D}}
\newcommand{\mcE}{\mathcal{E}}	\newcommand{\mcF}{\mathcal{F}}
\newcommand{\mcG}{\mathcal{G}}	\newcommand{\mcH}{\mathcal{H}}
\newcommand{\mcI}{\mathcal{I}}	\newcommand{\mcJ}{\mathcal{J}}
\newcommand{\mcK}{\mathcal{K}}	\newcommand{\mcL}{\mathcal{L}}
\newcommand{\mcM}{\mathcal{M}}	\newcommand{\mcN}{\mathcal{N}}
\newcommand{\mcO}{\mathcal{O}}	\newcommand{\mcP}{\mathcal{P}}
\newcommand{\mcQ}{\mathcal{Q}}	\newcommand{\mcR}{\mathcal{R}}
\newcommand{\mcS}{\mathcal{S}}	\newcommand{\mcT}{\mathcal{T}}
\newcommand{\mcU}{\mathcal{U}}	\newcommand{\mcV}{\mathcal{V}}
\newcommand{\mcW}{\mathcal{W}}	\newcommand{\mcX}{\mathcal{X}}
\newcommand{\mcY}{\mathcal{Y}}	\newcommand{\mcZ}{\mathcal{Z}}


%---------------------------------------
% Bold Math Fonts :-
%---------------------------------------

%Captital Letters
\newcommand{\bmA}{\boldsymbol{A}}	\newcommand{\bmB}{\boldsymbol{B}}
\newcommand{\bmC}{\boldsymbol{C}}	\newcommand{\bmD}{\boldsymbol{D}}
\newcommand{\bmE}{\boldsymbol{E}}	\newcommand{\bmF}{\boldsymbol{F}}
\newcommand{\bmG}{\boldsymbol{G}}	\newcommand{\bmH}{\boldsymbol{H}}
\newcommand{\bmI}{\boldsymbol{I}}	\newcommand{\bmJ}{\boldsymbol{J}}
\newcommand{\bmK}{\boldsymbol{K}}	\newcommand{\bmL}{\boldsymbol{L}}
\newcommand{\bmM}{\boldsymbol{M}}	\newcommand{\bmN}{\boldsymbol{N}}
\newcommand{\bmO}{\boldsymbol{O}}	\newcommand{\bmP}{\boldsymbol{P}}
\newcommand{\bmQ}{\boldsymbol{Q}}	\newcommand{\bmR}{\boldsymbol{R}}
\newcommand{\bmS}{\boldsymbol{S}}	\newcommand{\bmT}{\boldsymbol{T}}
\newcommand{\bmU}{\boldsymbol{U}}	\newcommand{\bmV}{\boldsymbol{V}}
\newcommand{\bmW}{\boldsymbol{W}}	\newcommand{\bmX}{\boldsymbol{X}}
\newcommand{\bmY}{\boldsymbol{Y}}	\newcommand{\bmZ}{\boldsymbol{Z}}
%Small Letters
\newcommand{\bma}{\boldsymbol{a}}	\newcommand{\bmb}{\boldsymbol{b}}
\newcommand{\bmc}{\boldsymbol{c}}	\newcommand{\bmd}{\boldsymbol{d}}
\newcommand{\bme}{\boldsymbol{e}}	\newcommand{\bmf}{\boldsymbol{f}}
\newcommand{\bmg}{\boldsymbol{g}}	\newcommand{\bmh}{\boldsymbol{h}}
\newcommand{\bmi}{\boldsymbol{i}}	\newcommand{\bmj}{\boldsymbol{j}}
\newcommand{\bmk}{\boldsymbol{k}}	\newcommand{\bml}{\boldsymbol{l}}
\newcommand{\bmm}{\boldsymbol{m}}	\newcommand{\bmn}{\boldsymbol{n}}
\newcommand{\bmo}{\boldsymbol{o}}	\newcommand{\bmp}{\boldsymbol{p}}
\newcommand{\bmq}{\boldsymbol{q}}	\newcommand{\bmr}{\boldsymbol{r}}
\newcommand{\bms}{\boldsymbol{s}}	\newcommand{\bmt}{\boldsymbol{t}}
\newcommand{\bmu}{\boldsymbol{u}}	\newcommand{\bmv}{\boldsymbol{v}}
\newcommand{\bmw}{\boldsymbol{w}}	\newcommand{\bmx}{\boldsymbol{x}}
\newcommand{\bmy}{\boldsymbol{y}}	\newcommand{\bmz}{\boldsymbol{z}}

\graphicspath{{./images/}}

\title{\huge{9I}}
\author{\huge{Jack Maguire}}
\date{}

\begin{document}
\maketitle

\part*{9I}

\qs{	
\begin{multicols}{2}
	\noindent
	\qsp{a}{
		\begin{align*}
			\differen &= 3x^2 - 6x + 1 \\
			\twodifferen &= 6x - 6 
		\end{align*}
		\begin{multicols}{2}
			\noindent
			\textbf{Convex}
			\begin{align*}
				6x - 6 &< 0 \\
				x &< 1
			\end{align*}	
			
			\columnbreak
			\textbf{Concave}
			\begin{align*}
				6x - 6 &> 0 \\
				x &> 1
			\end{align*}	
		\end{multicols}
	}
	\qsp{b}{
		\begin{align*}
			\differen &= 4x^3 - 9x^2 + 2 \\
			\twodifferen &= 12 x^2 - 18x
		\end{align*}
		\begin{multicols}{2}
			\noindent
			\textbf{Convex}
			\begin{align*}
				12x^2 - 18x &< 0 \\
				2x^2 - 3x &< 0 \\
				x(2x-3) &< 0 \\
				0 < x &< \frac{3}{2}
			\end{align*}	
			
			\columnbreak
			\textbf{Concave}
			\begin{align*}
				12x^2 - 18x &> 0 \\
				2x^2 - 3x &> 0 \\
				x(2x-3) &> 0 \\
				x < 0 \cup x &> \frac{3}{2}
			\end{align*}	
		\end{multicols}
	}
	\qsp{c}{
		\begin{align*}
			\differen &= \cos x \\
			\twodifferen &= - \sin x
		\end{align*}
		\begin{multicols}{2}
			\noindent
			\textbf{Convex}
			\begin{align*}
				- \sin x &< 0 \\
				0 < x &< \pi
			\end{align*}	
			
			\columnbreak
			\textbf{Concave}
			\begin{align*}
				- \sin x &> 0 \\
				\pi < x &> 2\pi
			\end{align*}	
		\end{multicols}
	}

	\columnbreak
	\qsp{d}{
		\begin{align*}
			\differen &= -2x + 3 \\
			\twodifferen &= -2
		\end{align*}
		\begin{multicols}{2}
			\noindent
			\textbf{Convex}
			\begin{align*}
				-2 &< 0 \\
				\text{Always Convex}
			\end{align*}	
			
			\columnbreak
			\textbf{Concave}
			\begin{align*}
				-2 &> 0 \\
				\text{Never Concave}
			\end{align*}	
		\end{multicols}
	}
	\qsp{e}{
		\begin{align*}
			\differen &= e^x - 2x \\
			\twodifferen &= e^x - 2
		\end{align*}
		\begin{multicols}{2}
			\noindent
			\textbf{Convex}
			\begin{align*}
				e^x - 2 &< 0 \\
				e^x &< 2 \\
				x &< \ln 2
			\end{align*}	
			
			\columnbreak
			\textbf{Concave}
			\begin{align*}
				e^x - 2 &> 0 \\
				e^x &> 2 \\
				x &> \ln 2
			\end{align*}	
		\end{multicols}
	}
	\qsp{f}{
		\begin{align*}
			\differen &= \frac{1}{x} \\
			\twodifferen &= \frac{-1}{x^2}
		\end{align*}
		\begin{multicols}{2}
			\noindent
			\textbf{Convex}
			\begin{align*}
				\frac{-1}{x^2} &< 0 \\
				-1 &< 0 \\
				\text{Always Convex}
			\end{align*}	
			
			\columnbreak
			\textbf{Concave}
			\begin{align*}
				\frac{-1}{x^2} &> 0 \\
				-1 &> 0 \\
				\text{Never Convex}
			\end{align*}	
		\end{multicols}
	}	
\end{multicols}
}

\qs{
\qsp{a}{
\[ \sin^2 y + \cos^2 y = 1 \quad \therefore \quad \cos y = \sqrt{1 - \sin^2 y} \]
\[ y = \arcsin x \quad \therefore \quad x = \sin y \]
\begin{align*}
	1 &= \cos y \differen \\
	\differen &= \frac{1}{\cos y} \\ 
	&= \frac{1}{\sqrt{1 - \sin^2 y}} \\
	&= \frac{1}{\sqrt{1 - x^2}}
\end{align*}
}
\qsp{b}{
\begin{align*}
	\differen &= \left( 1 - x^2 \right)^{-\frac{1}{2}} \\
	\twodifferen &= - \frac{1}{2} \left( 1-x^2 \right)^{-\frac{3}{2}} (-2x) \\
	&= \frac{x}{2 \sqrt{1-x^2}^3}
\end{align*}

\lineyboi

\[ 	\frac{x}{2 \sqrt{1-x^2}^3} > 0 \]
\begin{multicols}{2}
	\noindent
	\[ x > 0 \]
	\columnbreak
	\begin{align*}
		2 \sqrt{1-x^2}^3 &> 0 \\
		1 - x^2 &> 0 \\
		x^2 &< 1 \\
		-1 < x &< 1
	\end{align*}
\end{multicols}
\[ 0 < x < 1 \]
}
\qsp{d}{
\begin{align*}
	x &= 0 \\
	y &= \arcsin x = \arcsin 0 \\
	&= 0
\end{align*}
\[ = (0, 0) \]
}
}


\qs{
\qsp{a}{
\begin{align*}
	\differen &= -2 \cos x \sin x - 2 \cos x \\
	\twodifferen &= -2 \left( - \sin^2 x + \cos^2 x \right) + 2 \sin x \\
	&= -2 \left( \cos^2 x - \sin^2 x \right) + 2 \sin x \\
\end{align*}
\lineyboi
\begin{align*}
	0 &= -2 \left( \cos^2 x - \sin^2 x \right) + 2 \sin x \\
	0 &= \cos^2 x - \sin^2x  - \sin x \\
	0 &= 1 - \sin^2x - \sin^2x - \sin x \\
	0 &= 2 \sin^2x + \sin x - 1 \\
	0 &= \left( \sin x + 1 \right) \left( 2 \sin x - 1 \right) \\
	\sin x &= -1, \frac{1}{2} \\
	x &= \frac{3}{2}\pi, \frac{1}{6} \pi, \frac{5}{6} \pi
\end{align*}
\lineyboi
\begin{itemize}
	\ii \( (\frac{3}{2}\pi, 2) \)
	\ii \( (\frac{1}{6}\pi, -\frac{1}{4}) \)
	\ii \( (\frac{5}{6}\pi, -\frac{1}{4}) \)
\end{itemize}
}
\qsp{b}{
\begin{align*}
	\differen &= -\frac{\left( 3x^2 - 4x + 1 \right) \left( x - 2 \right) - \left( x^3 - 2x^2 + x - 1 \right) }{\left( x - 2 \right)^2} \\
	&= -\frac{3x^3 -6x^2 -4x^2 +8x +x -2 -x^3 +2x^2 -x +1}{\left( x - 2 \right)^2} \\
	&= \frac{-2x^3 +8x^2 -8x +1}{x^2 - 4x + 4} \\
	&= \frac{-2x^3 +8x^2 -8x}{x^2 - 4x + 4} + \frac{1}{x^2 - 4x + 4} \\
	&= \frac{-2x(x^2 - 4x + 4)}{x^2 + 4x + 4} + \frac{1}{x^2 - 4x + 4} \\
	&= -2x + \frac{1}{x^2 - 4x + 4} \\
	\twodifferen &= -2 -\left(x^2 - 4x + 4\right)^{-2}(2x-4) \\
	&= \frac{4-2x}{(x-2)^4} - 2 \\
\end{align*}
\lineyboi
\begin{align*}
	\frac{4-2x}{(x-2)^4} - 2 &= 0 \\
	\frac{4-2x}{(x-2)^4} &= 2 \\
	2-x &= (x-2)^4 \\
	-(x-2) &= (x-2)^4 \\
	-1 &= (x-2)^3 \\
	x-2 &= -1 \\
	x &= 1 \\
\end{align*}
\lineyboi
\[ = (1, 1) \]
}
\qsp{c}{
\begin{align*}
	\differen &= \frac{2x^2\left( x^2 - 4 \right) - x^3 \left( 2x \right)}{\left(x^2 - 4\right)^2} \\
	&= \frac{3x^4 - 8x^2 - 2x^4}{\left(x^2 - 4\right)^2} \\
	&= \frac{x^4-12x^2}{\left(x^2 - 4\right)^2} \\
	\twodifferen &= \frac{\left( 4x^3 - 24x \right)\left( x^2-4 \right)^2 - 2\left( x^4 - 12x^2 \right)\left( x^2 - 4 \right) (2x)}{\left(x^2 - 4\right)^4} \\
	&= \frac{\left( 4x^3 - 24x \right)\left( x^2-4 \right) - 4x\left( x^4 - 12x^2 \right)}{\left(x^2 - 4\right)^3} \\
	&= \frac{4x^5 - 16x^3 - 24x^3 + 96x - 4x^5 + 48x^3}{\left(x^2 - 4\right)^3} \\
	&= \frac{8x^3 + 96x}{\left(x^2 - 4\right)^3} \\
	&= \frac{8x\left( x^2 + 12 \right)}{\left(x^2 - 4\right)^3} \\
\end{align*}
\lineyboi
\begin{align*}
	0 = \frac{8x\left( x^2 + 12 \right)}{\left(x^2 - 4\right)^3}
	0 &= 8x\left( x^2 + 12 \right) \\
	x &= 0 \\ \\
	0 &= x^2 + 12 \\
	& \text{Discard}
\end{align*}
}
\lineyboi
\[ = (0, 0) \]
}


\qs{
\begin{align*}
	\differen &= \frac{2x^2}{x} + 4x \ln x \\
	&= 2x + 4x \ln x \\
	&= 2x \left( 1 + 2\ln x \right) \\
	\twodifferen &= 2\left( 1 + 2\ln x \right) + 2x \frac{2}{x} \\
	&= 2 + 4 \ln x + 4 \\
	&= 6 + 4 \ln x \\
\end{align*}
\lineyboi
\begin{align*}
	0 &= 6 + 4 \ln x \\
	-\frac{3}{2} &= \ln x \\
	x &= e^{-\frac{3}{2}} \\
	x & 0.223
\end{align*}
\[ = (0.223, -0.149) \]
}

\qs{
\qsp{a}{
\begin{align*}
	\differen &= e^x \left( x^2 - 2x + 2 \right) + e^x \left( 2x - 2 \right)
	&= e^x x^2
\end{align*}
\lineyboi
\[ 	0 = e^x x^2 \]
\begin{multicols}{2}
	\noindent
	\begin{align*}
		0 &= e^x \\
		x &= \ln 0 \\
		& \text{Discard}
	\end{align*}

	\columnbreak
	\begin{align*}
		0 &= x^2 \\
		x &= 0
	\end{align*}
\end{multicols}
\lineyboi
\[ = (0, 2) \]
\begin{tabular}{r|l}
	\( f^\prime(-0.1) \) & \( 0.09 \) \\
	\( f^\prime(0.1) \) & \( 0.011 \) \\
\end{tabular}
Point of Inflection?
}

\qsp{b}{
\[ \twodifferen = e^x \left( x^2 + 2x \right) = e^xx(x+2) \]
\begin{multicols}{3}
	\noindent
	\begin{align*}
		0 &= e^x \\
		x &= \ln 0 \\
		& \text{Discard}
	\end{align*}

	\columnbreak
	\[ x = 0 \]
	
	\columnbreak
	\begin{align*}
		0 &= x + 2 \\
		x &= -2 \\
	\end{align*}
\end{multicols}
Since \( x = 0 \) is a stationary point, I only need \( x = 2 \).
\[ = (2, 2e^2)\]
}
}

\qs{
\qsp{a}{
\[ \differen = e^x(x+1) \]
\lineyboi
\[ 0 = e^x(x+1) \]
\begin{multicols}{2}
	\noindent
	\begin{align*}
		0 &= e^x \\
		x &= \ln 0 \\
		& \text{Discard}
	\end{align*}

	\columnbreak
	\begin{align*}
		0 &= x + 1\\
		x &= -1 \\
	\end{align*}
\end{multicols}
\lineyboi
\[ = (-1, \frac{-1}{e}) \]
\begin{tabular}{r|l}
	\( f^\prime(-0.1) \) & \( -0.09 \) \\
	\( f^\prime(0.1) \) & \( 0.11 \) \\
\end{tabular}
Minimum Point
}
\qsp{b}{
\[ \twodifferen = e^x \left( x+1 + 1 \right) =  e^x(x+2) \]
\lineyboi
\[ 0 = e^x (x+2) \]
\begin{multicols}{2}
	\noindent
	\begin{align*}
		0 &= e^x \\
		x &= \ln 0 \\
		& \text{Discard}
	\end{align*}
	
	\columnbreak
	\begin{align*}
		0 &= x + 2\\
		x &= -2 \\
	\end{align*}
\end{multicols}
\lineyboi
\[ = (-2, \frac{-2}{e^2})) \]
}
\qsp{c}{
\vspace{5cm}
}
}

\end{document}